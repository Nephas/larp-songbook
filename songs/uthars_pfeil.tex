\begin{song}{title=Uthar's Pfeil, band=Traditional}
    \begin{verse}
        Der grimm Uthar mit seinem Pfeil \\
        tut nach dem Leben zielen. \\
        Sein Bogen schießt er ab mit Eil \\
        und läßt nicht mit sich spielen \\
        Das Leben schwindt \\
        wie Rauch im Wind, \\
        kein Fleisch mag ihm entrinnen. \\
        kein Gut noch Schatz \\
        findt bei ihm Platz: \\
        du mußt mit ihm von hinnen. \\
    \end{verse}

    \begin{verse}
        Kein Mensch auf Deren sagen kann, \\
        wann wir von hinnen gehen; \\
        wenn die schwarzen Schwingen nahn, \\
        dann hilft auch kein Flehen. \\
        Er nimmt mit G'walt \\
        hin Jung und Alt, \\
        tut sich vor niemand scheuen. \\
        Des Königs Stab \\
        bricht er bald ab \\
        und führt ihn an den Reihen. \\
    \end{verse}

    \begin{verse}
        Vielleicht ist heut der letzte Tag, \\
        den du noch hast zu leben. \\
        O Mensch, veracht nicht, was ich sag: \\
        nach Tugend sollst du streben! \\
        Wie mancher Mann \\
        wird müssen dran, \\
        so hofft noch viel der Jahren, \\
        und muss doch heint, \\
        weil d' Sonne scheint, \\
        zu den Hallen über fahren. \\
    \end{verse}

    \begin{verse}
        Der dieses Liedlein hat gemacht, \\
        von neuem hat gesungen, \\
        der hat gar oft den Tod betracht' \\
        und letztlich mit ihm 'rungen. \\
        Liegt jetzt im Hohl, \\
        es tut ihm wohl, \\
        tief in der Erd geborgen. \\
        Sieh auf dein Sach, \\
        du mußt hernach. \\
        es sei heute oder morgen. \\
    \end{verse}

\end{song}
