\begin{song}[remember-chords]{title=Über die Nebelberge, band=Traditional}
    \begin{multicols}{2}
        \begin{verse}
            ^{Em}Über die ^{D}Nebel - ^{Bm}berge ^{Em}weit, \\
            Zu Höhlen ^{D}tief aus ^{Bm}alter ^{Em}Zeit, \\
            Da ziehn wir ^{G}hin, da ^{D}lockt ^{Bm}Gewinn \\
            An Gold und ^{Em}Silber ^{D}und ^{Em}Geschmeid. \\
        \end{verse}

        \begin{verse}
            Wo ^einst das Reich der ^Zwer - ^ge ^lag, \\
            Wo ^glockengleich ihr ^Hammer - ^schlag \\
            Manch Wunder ^weckt, das ^still ^versteckt \\
            Schlief in ^Gewölben ^unter ^Tag. \\
        \end{verse}

        \begin{verse}
            Das ^Gold und Silber ^die - ^ser ^Erd \\
            Geschürft, ^geschmiedet ^und ^vermehrt. \\
            Sie fingen ^ein im ^edlen ^Stein \\
            Das Licht als ^Zierat ^für das ^Schwert. \\
        \end{verse}

        \columnbreak

        \begin{verse}
            An Silberkettchen Stern an Stern, \\
            Der Sonn- und Mondlichts reiner Kern, \\
            Von Drachenblut die letzte Glut \\
            Ging ein in Kronen großer Herrn. \\
        \end{verse}

        \begin{verse}
            Über die Nebelberge weit, \\
            Zu Höhlen tief aus alter Zeit, \\
            Dahin ich zieh in aller Früh \\
            Durch Wind und Wetter, Not und Leid. \\
        \end{verse}

        \begin{verse}
            Aus goldnen Bechern, ganz für sich, \\
            Da zechten sie allabendlich \\
            Bei Harfenklang und Chorgesang, \\
            Wo manche Stunde schnell verstrich. \\
        \end{verse}

        \begin{verse}
            Und knisternd im Gehölz erwacht \\
            Ein Brand. Von Winden angefacht, \\
            Zum Himmel rot die Flamme loht. \\
            Bergwald befackelte die Nacht. \\
        \end{verse}

        \begin{verse}
            Die Glocken läuteten im Tal, \\
            Die Menschen wurden stumm und fahl. \\
            Der große Wurm im Feuersturm \\
            Sengt’ ihre Länder schwarz und kahl. \\
        \end{verse}

        \begin{verse}
            Die Zwerge traf das Schicksal auch, \\
            Im Mondschein stand der Berg in Rauch. \\
            Durchs Tor entflohn, sanken sie schon \\
            Dahin in seinem Feuerhauch. \\
        \end{verse}

        \begin{verse}
            Über die Nebelberge hin \\
            Ins wilde Land lockt der Gewinn. \\
            Dort liegt bereits seit alter Zeit, \\
            Was unser war von Anbeginn. \\
        \end{verse}
    \end{multicols}
\end{song}